\section{Background and Motivation}
\label{sec:motivation}
The mysterious arrival of blockchain technology in 2008 with the creation of Bitcoin cryptocurrency and its subsequent adoption to the public in the coming years came to disrupt the perception and conceptualization of decentralized systems, financial independence and a new global era of reliability where open systems allow anyone to participate and contribute in exchange of incentives. A new economy emerged and thus new opportunities for the public and industries in private markets.

As time goes by Bitcoin keeps demonstrating its structural strength, self-governance, and democracy through general consensus. It has been unhackable ever since its creation whereas so many other companies and data breaches have happened.\cite{wikipedia_2022}. This "new way of working and cooperation" has led to rapid and astonishing advances in the field of cyber security, disrupting the economy and suddenly people with no access to credits or common financial instruments was able to exchange or transact assets through these systems and even more; blockchain system demonstrated to be the backbone and root of the new internet and new web, the sets of elements that would hep organizations and societies move into open systems able to be audited by anyone willing to gather information about them: the Distributed Ledger Technology.



THE IMPORTANCE OF BLOCKCHAIN IN THE INDUSTRY
PROMISES TO SOLVE ISSUES OF TIME AND TRUST SUPPLY CHAIN, FINANCE, LOGISTICS, HEALTHCARE

IMMUTABLE RECORD OF INTERACTION BETWEEN PARTNERS . CULPABILITY AND NON REPUDATION

PUBLIC = PERMISSIONLESS = CONSENSUS (PoW)


Shared ledger
Cryptography
Trust system or consensus
Smartcontractd (Business rules)

content addressing  

As many blockchain applications, solutions  and organizations are emerging nowadays, it comes to the realization that blockchain systems are here to stay, whereas many Specificity, complexity and uncertainty are some of the problems for its entire adoption. How to trust a blockchain system and its usability for an industry expecting to benefit from the system for the upcoming years without the need to migrate, have potential implications or errors related to the loss of trust of the application, maintainability or people no caring anymore about it.

As blockchain systems start being adopted by industries in the private sector, it becomes fundamental to create a trustful environment in which data can be securely shared but also avoid uploading corrupt data.
It becomes important to design a system that acknowledges the ownership of datasets that allow institutions its transmission and private reading through the emmission of access tokens. Also the creation of an NFT system can allow certain parties and institutions to cooperate. This thesis projects takes the work of \cite{holme2020secure} as a basis point to create the Ownership and non repudiation system through the creation of dataset \ac{NFT}'s.

WIth the help of Hyperledger fabric and Hyperledger fabric token SDK, it can be possible to integrate token generated systems and NFT to the forementioned Distributed computing system.

ERC-721
Long term support systems
How to trust?
How to use
How to make it easy for enterprises, 
How to make a single system broadly usable
Incentive system plays a very important role to attract the attention of the people and enterprises willing to use the system.

MLOps
- Experimentation and model development
- Deployment of updated models  into production
- Quality assurance


Data - Cleanup and shape
Version the source data and its attributes


Build the model  - Experimentation
- Feature selection
- Algorithm selection
- Hyperparameter tuning
- Fitting the model


Iterative process of experimentation

- Learn from mistakes
- Track metrics
- Source control the code
- Technology adoption lifecycle


\section{Objectives}

\begin{itemize}
\item Create a system able to represent ownership of certain digital assets through the emission of \ac{NFT}
\item Propose a system to transfer such NFTS between institutions as the equivalent of ownership transfer
\item Link the ownership system with the identity and blockchain databases.
\item Demonstrate that proposed functionality could enable institutions and corporations easy ways to cooperate and compute datasets by using ownership mechanisms
\end{itemize}


\section{Approach and Contributions}

\begin{itemize}
\item Give a brief summary of your overall approach.
\item Summarize the specific contributions that you made in this thesis (implementation, empirical results, analysis, etc.).
\end{itemize}


\section{Outline}

\begin{itemize}
\item Give an overview of the main points and the structure of your thesis.
\item Examples: ``Chapter 2 covers ...  Chapter 3 describes ...''
\item Show how the different parts (chapters) relate to each other.
\end{itemize}



IPFS DAG - Directed acyclic graph